
\documentclass{article}
\usepackage[utf8]{inputenc}


\usepackage[margin=1.1 in]{geometry}
\usepackage[T1]{fontenc}
\usepackage{mathtools}   % loads »amsmath«
\usepackage{amssymb}
\usepackage{amsfonts}
\usepackage{amsmath}
\usepackage{amsthm}
\usepackage{xcolor}
\usepackage{cancel}
%\usepackage{graphics}
\usepackage{graphicx}
%others
\usepackage{enumerate}
\usepackage{subcaption}



\usepackage{apacite}
\usepackage[round]{natbib}
%\bibliographystyle{plainnat}
\bibliographystyle{apacite}

\DeclarePairedDelimiter{\ceil}{\lceil}{\rceil}

\setlength{\parskip}{0.8em}
\usepackage{setspace}
%\singlespacing
\spacing{1.15}



\newtheorem{defin}{Definition.}
\newtheorem{teo}{Theorem. }
\newtheorem{lema}{Lemma. }
\newtheorem{coro}{Corolary. }
\newtheorem{prop}{Proposition. }
\theoremstyle{definition}
\newtheorem{examp}{Example. }
\newtheorem{problem}{}

\title{Metrics, PS5}
\author{Giselle Labrador Badia}
\date{May 2022}

\begin{document}

\maketitle

 Tables with standard errors are provided for all regressions and other relevant analyses that I discuss. All the Julia code pertinent to this assignment is attached. 
 
 \subsection*{Part 1: Analytic exercises}
 
 \noindent \hspace{0.41cm} \textbf{Question 1.}  Assume a returns to schooling model with just 1 unobserved variable ability $A$ distributed $U[0,1]$. Our potential outcomes model for earnings by schooling (treated) or not is
$$
\begin{gathered}
Y_{1}=1+0.5 A \\
Y_{0}=A
\end{gathered}
$$
Treatment/schooling is determined by
$$
D=1\{-0.5+A>0\}
$$
 
 \begin{itemize}
     \item[(a)] Average treatment effect  is $$ATE = \mathbb{E}[Y_{1}  - Y_{0}]= \mathbb{E}[1+0.5A- A]= \mathbb{E}[1- 0.5 A] = 1 - 0.5 \mathbb{E}[A] = 0.75$$
     
    %See that ATE can also be defined in terms of MTE as $ATE = \mathbb{E}[MTE] =  \mathbb{E}[Y_{1} - Y_0]$
    
    \item[(b)] The fraction of the population that takes the treatment is 
    $$ \mathbb{P}[D=1] = \mathbb{P}[-0.5+A>0] = \mathbb{P}[A>0.5] = 1- \mathbb{P}[A<0.5] = 0.5 $$
    
     \item[(c)] The maximum treatment effect is
     
     $$\max_{A\in[0,1]} TE = \max_{A\in[0,1]} (Y_{1} - Y_0) = \max_{A\in[0,1]} (1-0.5A) = 0,\qquad \text{ attained at }A=0.$$
     
     The minimum treatment effect is
          $$\min_{A\in[0,1]} TE = \min_{A\in[0,1]} (Y_{1} - Y_0) = \min_{A\in[0,1]} (1-0.5A) = 0.5,\qquad \text{ attained at }A=1.$$
          
     \item[(d)] Assume A distributed Normal, $A \sim N(0,1)$. Now the support in unbouded. 
     
     The maximum treatment effect is
     
     $$\sup_{A\in[-\infty,\infty]} TE = \infty$
     
     The minimum treatment effect is
          $$\inf_{A\in[-\infty,\infty]} TE = -\infty$$
      
      \item[(e)] The average treatment effect on the treated is
      \begin{align*}
          ATET &= \mathbb{E}[Y_1-Y_0|D=1] = \mathbb{E}[1-0.5A|D=1]\\ 
          &= \mathbb{E}[1-0.5A|A>0.5] =1-0.5\mathbb{E}[A|A>0.5]  \\
          &= 1-0.5*0.75 = 0.625
      \end{align*}
      The average treatment effect on the untreated is
           \begin{align*}
          ATEU &= \mathbb{E}[Y_1-Y_0|D=0] = \mathbb{E}[1-0.5A|D=0]\\ 
          &= \mathbb{E}[1-0.5A|A<0.5] =1-0.5\mathbb{E}[A|A<0.5]  \\
          &= 1-0.5*0.25 = 0.875
      \end{align*}
      $$$$
       
    \item[(f)] ATEU > ATET because $TE = 1-0.5A$ is decreasing in $A$; in words, education (treatment) will have a greater effect on the earnings of individuals with low ability. Individuals with high ability will have relatively high earnings with or without education. 
    
    \item[(g)] The OLS estimand for the effect of $D$ on $Y$ is
    \begin{align*}
        \beta(OLS) &= \mathbb{E}[Y|D=1] - \mathbb{E}[Y|D=0]\\
                   &= \mathbb{E}[1+0.5A|A>0.5] - \mathbb{E}[A|A<0.5]\\
                   &= 1 + 0.5*0.75  - 0.5\\
                   &= 1.125
    \end{align*}
    
    \item[(h)] The OLS estimand is biased upward for the ATE because treatment is not random but biased towards those with higher ability. More formally, selection into treatment requires a higher ability, thus conditional independence fails.

   \end{itemize}
   
    \noindent \hspace{0.41cm} \textbf{Question 2.}  2) Assume the potential outcomes model with $V=\delta_{0}+\delta_{1} Z+U_{V}$, for instrument $Z \in\{0,1\}$.
    \begin{itemize}
        \item[(a)] Now I prove that this model implies the Angrist and Imbens monotonicity assumption. (Note:  I omitted $i$ subscripts, but $\delta_{0}$ and $\delta_{1}$ are homogeneous parameters.)
        
        Monotonicity implies that given parameters $\delta_0$, $\delta_1$, $V_i(Z)$ is monotonic in $Z$ for all $i$.
        
        See that $V_i(Z=1)-V_i(Z=0) = \delta_1$ for all i, which means
        \begin{itemize}
            \item $V_i(Z=1)>V_i(Z=0)$ if $\delta_1>0$ $\Rightarrow$ $V_i$ is increasing in $Z$
            \item $V_i(Z=1)>V_i(Z=0)$ if $\delta_1<0$ $\Rightarrow$ $V_i$ is decreasing in $Z$
            \item $V_i(Z=1)>V_i(Z=0)$ if $\delta_1=0$ $\Rightarrow$ $V_i$ does not change with $Z$
        \end{itemize}
        
        \item[(b)] In this model, a new function for $V$ such that monotonicity does not hold is
        $$ V_i = \delta_{0}+\delta_{i1} Z +U_{iV}, \text{, where } \delta_{i1} \in \{-1,1\}$$
        This model presents heterogeneous treatment effects, now $V_i(Z=1)-V_i(Z=0) = \delta_{i1}$ could be 1 or -1. Hence, monotonicity does not hold. 
    \end{itemize}
    
     
    \noindent \hspace{0.41cm} \textbf{Question 3.}  Assume $U_{V}$ is distributed Uniform $[-2,2]$, and $V=Z+U_{V}$ with $Z \in$ $\{0,1\}$. As usual, I assume individuals take treatment when $V>0$.
    \begin{itemize}
        \item[(a)] The range of $U_{V}$ values for the complier, defier, always taker, and never taker groups is
        \begin{itemize}
            \item Compliers ($C$): Take treatment when $Z=1$ and do not take the treatment when $Z=0$.
            $$\iff V_1>0 \text{ and } V_0<0 \iff -1 \leq U_V \leq 0 $$
             \item Defiers ($D$): Take treatment when $Z=0$ and do not take the treatment when $Z=1$.
            $$\iff V_0>0 \text{ and } V_1<0 \iff -1 \geq U_V { and } U_V \leq 0 $$
            which implies that there are no defiers. From the monotoniity assumption we can also infer that there will not be defiers.
            \item Always takers ($AT$): Take treatment when $Z=1$ and when $Z=0$.
            $$\iff V_1>0 \text{ and } V_0>0 \iff U_V \geq 0 $$
            \item Never takers ($NT$): Do not take treatment when $Z=1$ and when $Z=0$.
            $$\iff V_1<0 \text{ and } V_0<0 \iff U_V \leq -1 $$
            
            
        \end{itemize}
\item[(b)] Hence, the fraction of the population in each group.

        \begin{itemize}
            \item Compliers ($C$): $\mathbb{P}(-1 \leq U_V \leq 0 ) = \frac{1}{4}$
             \item Defiers ($D$): $\mathbb{P}(D) = 0$
            \item Always takers ($AT$): $\mathbb{P}(U_V\leq -1) = \frac{1}{4}$
            \item Never takers ($NT$):  $\mathbb{P}( U_V\geq 0 ) = \frac{1}{2}$
            
    \end{itemize}
    
          
    \noindent \hspace{0.41cm} \textbf{Question 4.}  Assume there are 2 types in the population. Type 1 has treatment effect $\Delta=2$ and Type 2 has $\Delta=-1$. 30 percent of the population is Type 1 , 70 percent Type 2. Type 1 s have utility given by $V=Z+U_{V}$, with $U_{V} \sim$ $U[-1,1]$ and Type 2 s have utility given by $V=2 Z+U_{V}$, with $U_{V} \sim U[-1,1]$. Let the instrument $Z \in\{0,1\}$ and $\operatorname{Pr}(Z=1)=0.5$.
               
        \begin{itemize}
        \item[(a)] ATE is
        \begin{align*}
             ATE &= \mathbb{E}[\Delta] = \mathbb{P}(Type 1)\mathbb{E}[\Delta|Type 1] + \mathbb{P}(Type 2)\mathbb{E}[\Delta|Type 2]\\
             &=0.3*2+0.7*(-1) = -0.1
        \end{itemize}
        \item[(b)] The probability of being treated given the instrument $Z$ is
                \begin{align*}
             \mathbb{P}(D=1|Z=1) &= \mathbb{P}(Type 1)\mathbb{P}(U_V>-1) + \mathbb{P}(Type 2)\mathbb{P}(U_V>-2)\\
             &=0.3*1+0.7*1 = 1
        \end{itemize}
            \begin{align*}
             \mathbb{P}(D=1|Z=0) &= \mathbb{P}(Type 1)\mathbb{P}(U_V>0) + \mathbb{P}(Type 2)\mathbb{P}(U_V>0)\\
             &=0.3*\frac{1}{2}+0.7*\frac{1}{2} = \frac{1}{2}
        \end{itemize}
        \item[(c)] LATE is same as ATE. See that $Z$ and $U_V$ doe not affect the treatment effect of any of the two groups. Compliers are $U_V \leq 0$ for the two types. More formally:
                        \begin{align*}
             LATE &= \mathbb{E}[\Delta|U_V<0] = \mathbb{P}(Type 1)\mathbb{E}[\Delta|Type 1,U_V<0] + \mathbb{P}(Type 2)\mathbb{E}[\Delta|Type 2,U_V<0]\\
             &= \mathbb{P}(Type 1)\mathbb{E}[\Delta|Type 1] + \mathbb{P}(Type 2)\mathbb{E}[\Delta|Type 2]\\
             &=ATE = -0.1 
        \end{itemize}
        
        \end{itemize}
 \subsection*{Part 2: Monte Carlo exercises}
 
    \noindent \hspace{0.41cm} \textbf{Question 1.}  
    Assume log hourly earnings for individual $i$ takes this form:
$$
\ln w_{i}=1+0.05 s_{i}+0.1 a_{i}+\epsilon_{i}
$$
where $s_{i}$ is units of observed schooling and $a_{i}$ is unobserved ability. $\epsilon_{i} \sim$ $N(0,0.5) . a_{i} \sim N(0,4)$
Units of schooling for each individual $i$ are
$$
s_{i}=3 a_{i}+z_{i 1}+z_{i 2}+\eta_{i}
$$
where $\eta_{i} \sim N(0,1) . z_{i 1}$ and $z_{i 2}$ reflect the cost of schooling for individual $i$. $z_{i 1} \sim N(0,0.1) . z_{i 2} \sim N(0,25)$. There is another variable $z_{3 i}$ which is distributed Uniform $[0,1] . z_{1 i}, z_{2 i}, z_{3 i}$ are all independent of each other, and independent of $\eta_{i}, a_{i}, \epsilon_{i}$.

\textit{Solution:}
The code that simulates data is in $Model1.jl$. For question b) you can refer to table 1 in the table below. 

The instrument $z_1$ affects wages, so it is a relevant instrument. , though has a very small variance so it is a weak instrument. The instrument $z_2$ is relevant and it is a strong instrument with a lot of variation. The instrument $z_3$ is not relevant, it does not have any impact on wages. 
All instruments are exogenous, that is, if anything, they do not affect wages through any variable other than schooling. 

The schooling coefficient, $\beta_2$ is biased upwards, but it is pretty close taking into account that there is endogeneity. The explanation above about the instruments explain many of the results we see in the table. The coefficient of the iv with the first instrument is not significant which is expected, and the coefficient is very far from the true effect. Surprisingly, the results with $z_3$ are not as far from the real parameter (half) and it is significant. Anything that involves the valid instrument $z_2$ is a good estimate. Looking at the F statistic, we see that the strong set of instruments is those that include $z_2$. The highest F-statistics is attained with only the second instrument. Similar results but with better estimates are obtained by repeating the exercise with more observations. 
    \begin{table}
    \centering
\begin{tabular}{r|rrrrrrrr}
\hline
 & OLS & IV1 & IV2 & IV3 & IV12 & IV13 & IV23 & IV123\\
\hline
N=2000 & & & & & & & & \\

 & 0.0001915 & 39.6 & 0.0002082 & 0.0005187 & 0.000208 & 0.000521 & 0.0002082 & 0.000208\\
intercept & 0.9778 & 0.8368 & 0.9773 & 0.9758 & 0.9773 & 0.9758 & 0.9773 & 0.9773\\
 & 2.705e-7 & 8644.0 & 3.698e-7 & 0.0006591 & 3.691e-7 & 0.000662 & 3.697e-7 & 3.691e-7\\
schooling & 0.05616 & -2.07 & 0.04932 & 0.02591 & 0.04936 & 0.02581 & 0.04932 & 0.04936\\
F stat & - & 0.0005234 & 7769.0 & 2.213 & 3901.0 & 1.106 & 3883.0 & 2599.0\\
\hline
N=50000 & & & & & & & & \\
 & 7.608e-7 & 7.96e-7 & 8.204e-7 & 4.381e-5 & 8.204e-7 & 8.022e-7 & 8.204e-7 & 8.204e-7\\
intercept & 1.001 & 1.001 & 1.0 & 1.002 & 1.0 & 1.001 & 1.0 & 1.0\\
 & 9.904e-10 & 3.996e-5 & 1.316e-9 & 0.05244 & 1.316e-9 & 3.99e-5 & 1.316e-9 & 1.316e-9\\
schooling & 0.05624 & 0.05871 & 0.05001 & 0.1335 & 0.05001 & 0.05942 & 0.05001 & 0.05001\\
F stat & - & 12.55 & 2.155e6 & 0.1233 & 1.078e6 & 6.332 & 1.077e6 & 718400.0\\
\hline

\end{tabular}
\end{table}

\vspace{0.2cm}
\noindent \hspace{0.41cm} \textbf{Question 2.} 
    
Consider the following potential outcomes model:
$$
\begin{gathered}
Y_{0}=1+U_{0} \\
Y_{1}=4+U_{1} \\
V=-1+2 Z+U_{V}
\end{gathered}
$$
with $\left(U_{0}, U_{1}, U_{V}\right) \sim N(0, \Sigma)$, and $\Sigma$ corresponding to the following variancecovariance elements: $V\left(U_{0}\right)=1, V\left(U_{1}\right)=1, V\left(U_{V}\right)=1, \operatorname{Cov}\left(U_{0}, U_{1}\right)=0.5$, $\operatorname{Cov}\left(U_{0}, U_{V}\right)=0.3$, and $\operatorname{Cov}\left(U_{1}, U_{V}\right)=0.7$. $Z$ is a valid instrument with $Z \in\{0,1\}$, and $\operatorname{Pr}(Z=1)=0.3$.


\textit{Solution:}
The code that simulates data is in $Model2.jl$ (a).

Below I show a table with the solutions to questions (b), (c), and (d). 

\begin{table}[h]
\centering
\begin{tabular}{lll}
\hline
             & b) & d) \\ \hline
ATE          &  2.951  & 2.951   \\ 
ATET         &  3.176  & 3.131   \\
ATEU         &  2.826  &  2.831  \\ 
OLS          &  3.592   & 3.434   \\
ITT          &  1.979 &  2.356  \\ 
IV           & 2.950 &  2.915  \\ 
\% compliers c) & 0.69   &  0.82   \\ \hline
\end{tabular}
\end{table}


    
\end{document}